\section{Reproducibility and Open Science}

\begin{itemize}
    \item Code and configs released under MIT license at \href{https://github.com/ConceptualFragmentationInLLMsAnalysisAndVisualization}{GitHub repository}
    \item Seed lists and hyperparameters logged in JSON format
    \item Dockerfile ensures environment parity across research teams
    \item Negative results and failed variants documented in appendices
    \item LLM prompts and responses cached for reproducibility
\end{itemize}

\paragraph{Key pipeline steps (pseudocode).}

\begin{verbatim}
# 1. Train baseline model and cache activations
python train_baseline.py --dataset <dataset> --seed <seed>

# 2. Compute cluster paths and metrics
python concept_fragmentation/analysis/cluster_paths.py \
       --dataset <dataset> --seed <seed> --compute_similarity

# 3. Generate cluster labels and LLM narratives
python llm_path_analysis.py --dataset <dataset> --seed <seed>

# 4. Build LaTeX fragments & figures
python tools/build_paper_tables.py
python generate_paper_figures.py --dataset <dataset>
\end{verbatim}

Full, runnable code is available in the public repository; all prompts and
LLM responses are cached for deterministic builds.

\subsection{LLM Prompts for Cluster Interpretation}

To ensure reproducibility of our LLM-powered analysis, we document the key prompts used for cluster interpretation and path analysis:

\paragraph{Cluster Labeling Prompt:}
\begin{verbatim}
You are analyzing clusters from a neural network. 
For cluster L{layer}_C{cluster} containing these words:
{sample_words}

Category distribution: {category_counts}
Cluster size: {size} words

Provide a concise, interpretable label that captures 
the semantic or grammatical essence of this cluster.
\end{verbatim}

\paragraph{Path Narrative Prompt:}
\begin{verbatim}
Analyze this concept trajectory through GPT-2:
Path: {path}
Window: {window_name}
Grammatical distribution: {grammatical_counts}

Explain how concepts evolve through these clusters,
focusing on the transformation from semantic to 
grammatical organization.
\end{verbatim}

\paragraph{Bias Analysis Prompt:}
\begin{verbatim}
Analyze potential biases in these neural pathways:
Path: {path}
Demographics: {demographic_stats}
Outcome distribution: {outcomes}

Identify any concerning patterns or biases in how
different demographic groups are processed.
\end{verbatim}

Interactive demos and full code implementation are available on our project repository.