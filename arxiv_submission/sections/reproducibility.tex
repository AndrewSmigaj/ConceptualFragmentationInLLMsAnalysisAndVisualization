\section{Reproducibility and Open Science}

\begin{itemize}
    \item Code and configs released under MIT license at \href{https://github.com/ConceptualFragmentationInLLMsAnalysisAndVisualization}{GitHub repository}
    \item Seed lists and hyperparameters logged in JSON format
    \item Dockerfile ensures environment parity across research teams
    \item Negative results and failed variants documented in appendices
    \item LLM prompts and responses cached for reproducibility
\end{itemize}

\paragraph{Key pipeline steps (pseudocode).}

\begin{verbatim}
# 1. Train baseline model and cache activations
python train_baseline.py --dataset <dataset> --seed <seed>

# 2. Compute cluster paths and metrics
python concept_fragmentation/analysis/cluster_paths.py \
       --dataset <dataset> --seed <seed> --compute_similarity

# 3. Generate cluster labels and LLM narratives
python llm_path_analysis.py --dataset <dataset> --seed <seed>

# 4. Build LaTeX fragments & figures
python tools/build_paper_tables.py
python generate_paper_figures.py --dataset <dataset>
\end{verbatim}

Full, runnable code is available in the public repository; all prompts and
LLM responses are cached for deterministic builds.

Interactive demos and full code implementation are available on our project repository.

\begin{figure}[ht]
    \centering
    \includegraphics[width=0.8\textwidth]{figures/subspace_angle.png}
    \caption{Subspace angle analysis showing convergence and divergence patterns across layers. Lower angles represent more coherent trajectories.}
    \label{fig:subspace_angle}
\end{figure}