\section{Reproducibility and Open Science}

\begin{itemize}
    \item Code and configs released under MIT license at \href{https://github.com/AndrewSmigaj/conceptual-trajectory-analysis-LLM-intereptability-framework/}{GitHub repository}
    \item Seed lists and hyperparameters logged in JSON format
    \item Python requirements.txt files ensure environment reproducibility
    \item Negative results and failed variants documented in appendices
    \item LLM prompts and responses cached for reproducibility
\end{itemize}

LLM responses are cached for deterministic builds.

\subsection{LLM Prompts for Cluster Interpretation}

To ensure reproducibility of our LLM-powered analysis, we document the key prompts used for cluster interpretation and path analysis:

\paragraph{Cluster Labeling Prompt:}
\begin{verbatim}
You are analyzing clusters from a neural network. 
For cluster L{layer}_C{cluster} containing these words:
{sample_words}

Category distribution: {category_counts}
Cluster size: {size} words

Provide a concise, interpretable label that captures 
the semantic or grammatical essence of this cluster.
\end{verbatim}

\paragraph{Path Narrative Prompt:}
\begin{verbatim}
Analyze this concept trajectory through GPT-2:
Path: {path}
Window: {window_name}
Grammatical distribution: {grammatical_counts}

Explain how concepts evolve through these clusters,
focusing on the transformation from semantic to 
grammatical organization.
\end{verbatim}

\paragraph{Bias Analysis Prompt:}
\begin{verbatim}
Analyze potential biases in these neural pathways:
Path: {path}
Demographics: {demographic_stats}
Outcome distribution: {outcomes}

Identify any concerning patterns or biases in how
different demographic groups are processed.
\end{verbatim}

Full code implementation and example scripts are available on our project repository.