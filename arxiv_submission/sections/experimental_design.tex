\section{Experimental Design}

Our experiments validate CTA across diverse domains, from medical AI to language understanding:

\subsection{Datasets and Models}

\begin{itemize}
    \item \textbf{Heart Disease Diagnosis}: UCI Heart Disease dataset (303 patients, 13 clinical features) with 3-layer MLP, demonstrating medical AI interpretability
    \item \textbf{GPT-2 Semantic Subtypes}: 1,228 validated single-token words across 8 semantic categories, analyzed through GPT-2's 12 layers (embedding layer + 11 transformer blocks, 117M parameters)\footnote{We follow standard convention in numbering GPT-2's layers: Layer 0 is the embedding layer, and Layers 1-11 correspond to transformer blocks 1-11. The 12th transformer block was not analyzed in this study.}
    \item \textbf{GPT-2 Semantic Pivot}: 202 sentences with contradictory information, tracking semantic processing
\end{itemize}

\subsection{Unified CTA Methodology}

We employ a three-phase approach:

\begin{enumerate}
    \item \textbf{Optimal Clustering}: Gap statistic determines layer-specific cluster counts (e.g., k=4 for GPT-2 layer 0, k=2 for layers 1-11)
    \item \textbf{Unique Labeling}: Every cluster receives globally unique ID (L\{layer\}\_C\{cluster\}) preventing cross-layer confusion
    \item \textbf{Windowed Analysis}: Temporal segmentation reveals phase transitions (see Section 3.5)
\end{enumerate}

\subsection{Metrics and Validation}

\begin{itemize}
    \item \textbf{Cross-layer Metrics}: Centroid similarity ($\rho^c$), membership overlap ($J$), trajectory fragmentation (F, see Section 3.6)
    \item \textbf{Stability Analysis}: Window-based stability scores revealing reorganization points
    \item \textbf{Path Statistics}: Convergence ratios, diversity indices, archetype identification
    \item \textbf{Clinical Validation}: For heart disease, correlation of fragmentation with diagnostic uncertainty
\end{itemize}

\subsection{Visualization Suite}

\subsubsection{Sankey Diagram Visualization Scheme}
\label{sec:sankey_scheme}

Throughout this paper, we employ Sankey diagrams as our primary visualization for concept trajectories through neural networks. In our standardized scheme:

\begin{itemize}
    \item \textbf{Nodes}: Represent clusters at each layer, labeled with unique identifiers (e.g., L4\_C1) and semantic interpretations
    \item \textbf{Links}: Show the flow of datapoints between clusters across consecutive layers
    \item \textbf{Link Width}: Proportional to the number of datapoints following that path
    \item \textbf{Colors}: Distinguish archetypal pathways or semantic categories
    \item \textbf{Vertical Layout}: Layers progress from left (input) to right (output), providing intuitive flow visualization
\end{itemize}

This visualization enables immediate comprehension of how concepts reorganize through network layers, revealing convergence patterns, phase transitions, and dominant processing pathways.

\subsubsection{Implementation Tools}

\begin{itemize}
    \item \textbf{Concept MRI Tool}: Software implementing CTA with interactive Sankey diagrams
    \item \textbf{Clinical Dashboards}: Patient archetype visualization for medical interpretability
    \item \textbf{Interactive Exploration}: Web-based interfaces for real-time analysis
\end{itemize}


