\section*{LLM-Generated Neural Network Path Narratives}

\subsection*{Dataset: Titanic}

This report presents LLM-generated narratives that interpret the archetypal paths
identified in our neural network analysis. Each narrative translates technical
patterns into an accessible explanation of the model's internal logic.

\subsection*{Archetypal Path 1: L0C3→L1C0→L2C1→L3C0}

<details>
<summary>Path Statistics</summary>

\begin{verbatim}
Path: L0C3→L1C0→L2C1→L3C0
Count: 209 passengers (23.46\\% of dataset)
\end{verbatim}
</details>

\subsubsection*{LLM Narrative}

Imagine the neural network as a ship captain, attempting to predict who among the passengers will survive the Titanic disaster. Each layer of the network represents a different decision-making stage, with the path through the layers showing the captain's thought process.

The journey begins at L0C3, where the captain considers all passengers indiscriminately. As he moves to L1C0, he starts to focus on specific features such as class and fare – perhaps noticing that wealthier passengers have a higher chance of survival.

At L2C1, the captain picks up on finer details. Maybe he notices that within the first-class passengers, those who are younger or travelling without family are more likely to survive. This is the captain learning to pick out specific patterns and correlations within the chaos.

Finally, when he reaches L3C0, the captain makes his prediction. He has taken all the information into account – class, fare, age, family status – and arrived at a decision. This path through the network, then, is the captain's journey from considering all passengers equally to making a calculated prediction about their survival.

However, the captain must be careful. If he relies too much on certain features such as class or fare, he could unfairly favor the wealthier passengers. To avoid this, he should consider a broader range of factors and ensure his decision-making process is as fair as possible.

---

\subsection*{Archetypal Path 2: L0C1→L1C0→L2C1→L3C0}

<details>
<summary>Path Statistics</summary>

\begin{verbatim}
Path: L0C1→L1C0→L2C1→L3C0
Count: 93 passengers (10.44\\% of dataset)
\end{verbatim}
</details>

\subsubsection*{LLM Narrative}

Imagine for a moment that our neural network is a series of interconnected rooms. Each room represents a cluster and the doors between them represent the transitions the model makes as it processes information.

Our journey starts in Room L0C1, a bustling, diverse space filled with raw, unprocessed data. Here, the model sifts through a myriad of features, from passenger age and gender to cabin class and embarkation point.

As we step through the door to Room L1C0, the noise and confusion of L0C1 fades. The model has isolated key features and interactions, leaving behind the less relevant information. It's a more focused, streamlined environment.

Our next transition takes us to Room L2C1. This room is even more refined, with the model having further narrowed down the most predictive features. The atmosphere is tense, as the model closes in on its final decision.

Finally, we enter Room L3C0. This is where the model converges to a decision. The room is calm and decisive, reflecting the model's certainty about its prediction.

This journey through the model's latent space reveals a process of gradual refinement and focus. The model starts with a broad, chaotic set of features, but through a series of transformations, it narrows its focus to the elements most predictive of the outcome. This path is a testament to the model's ability to distill complexity into simplicity, to find patterns in a sea of noise.

---

\subsection*{Archetypal Path 3: L0C4→L1C0→L2C1→L3C0}

<details>
<summary>Path Statistics</summary>

\begin{verbatim}
Path: L0C4→L1C0→L2C1→L3C0
Count: 83 passengers (9.32\\% of dataset)
\end{verbatim}
</details>

\subsubsection*{LLM Narrative}

(40\%)

Embark on a journey through the neural network's internal universe, traversing from the initial layer L0C4, through L1C0 and L2C1, and finally arriving at L3C0. This journey is one of discovery and decision-making, mirroring the harrowing voyage of the Titanic's passengers.

Beginning in the vast expanse of L0C4, the neural network gazes upon the raw, unprocessed data. This layer is like the initial boarding of the Titanic, with passengers from all walks of life coming together, each carrying unique characteristics and experiences.

As we move to L1C0, the network begins to decipher patterns from the chaos. It's as if the ship's crew is starting to organize passengers based on certain criteria: first-class or third-class tickets, traveling alone or with family, young or old. These criteria are the features that the model identifies and uses to divide the passengers into different groups.

The journey continues to L2C1, where the network refines its understanding further. It's akin to the crew observing more nuanced interactions between passengers, such as a wealthy elderly couple or a young woman traveling alone in third class. Here, the model learns more complex relationships between features, adding depth to its understanding.

Finally, we arrive at L3C0, the output layer. This is the moment of truth, the life-or-death prediction. The model, like the ship's crew during the disaster, must make a decision based on the information it has gathered. Just like the real-world event, the model's decisions are influenced by the patterns it has learned, reflecting the complex interplay of factors that determined survival on the Titanic.

This journey through the latent space of the neural network provides a window into the model's decision-making logic, offering a glimpse of how it learns and evolves from raw data to refined predictions.

---

\subsection*{Archetypal Path 4: L0C2→L1C1→L2C0→L3C1}

<details>
<summary>Path Statistics</summary>

\begin{verbatim}
Path: L0C2→L1C1→L2C0→L3C1
Count: 81 passengers (9.09\\% of dataset)
\end{verbatim}
</details>

\subsubsection*{LLM Narrative}

\subsection*{Technical Insights}

The path L0C2$\rightarrow$L1C1$\rightarrow$L2C0$\rightarrow$L3C1 represents the journey of 81 passengers (approximately 9.09\% of the dataset) through the neural network. This path's structure is indicative of the model's internal representations and its learning process. 

The transition from L0C2 to L1C1 suggests the model is refining its understanding of the data. L0C2 could be identifying rudimentary features while L1C1 may be further refining these features. 

The transition from L1C1 to L2C0 suggests further convergence on a specific set of features or attributes. And finally, transition from L2C0 to L3C1 signifies that the model is converging on a final decision, based on the features it has learned so far. 

This path exhibits a pattern of convergence, as it starts from an undefined cluster (L0C2) and moves through the layers, gradually refining the features and narrowing down to a final decision in L3C1.

\subsection*{Fairness Assessment}

Without detailed labels or demographic information for each cluster, it is challenging to determine whether this path exhibits disparate treatment across demographic groups. However, given the model was trained on the Titanic dataset, potential biases could stem from societal structures at the time, such as socio-economic status, gender, and age. 

If the model exhibits bias, potential interventions include modifying the training data to ensure balanced representation of different groups, applying regularization techniques to reduce overfitting, or using fairness-oriented learning algorithms that take into account the potential biases in the data.

\subsection*{Narrative Synthesis}

Think of this path as a traveler’s journey, starting from a broad, undefined territory (L0C2) and moving towards a final destination (L3C1). Along the journey, the traveler stops at various checkpoints (L1C1, L2C0) where they gain more specific information that refines their understanding of the destination.

The first leg of the journey from L0C2 to L1C1 is like moving from an open field into a winding forest trail. The model is refining its understanding of the broad features, focusing on certain patterns while discarding others.

Next, the traveler moves from the forest trail (L1C1) to a narrow mountain path (L2C0). The model is now focusing on more specific features, converging on a smaller set of outputs.

Finally, the traveler reaches their destination at L3C1. The model has now converged on a final decision. It has learned the path through the layers of the network, refining its understanding at each step, and arriving at a decision influenced by its journey.

This path through the neural network reveals that the model's decision-making logic is a process of refinement and convergence from broad, undefined clusters to more specific ones. It’s like a traveler moving from an open field to a defined destination, learning and refining their path along the way.

---

\subsection*{Archetypal Path 5: L0C0→L1C0→L2C1→L3C0}

<details>
<summary>Path Statistics</summary>

\begin{verbatim}
Path: L0C0→L1C0→L2C1→L3C0
Count: 76 passengers (8.53\\% of dataset)
\end{verbatim}
</details>

\subsubsection*{LLM Narrative}

Imagine the neural network as a complex system of waterways. The data (passengers) enters this system at L0C0, like a mighty river flowing into a vast delta. This is the raw, unprocessed data, carrying a diverse array of passenger information.

As the data flows into the first layer (L1C0), the delta's channels filter and sort the data, separating out important features and discarding irrelevant noise. This is like the river distributing its water and sediment across the delta, creating distinct patterns and channels.

At the second layer (L2C1), the filtered data converges into a smaller number of channels. These channels represent more complex feature interactions. It's as if smaller streams in the delta have merged into larger, more powerful rivers, carrying an integrated mix of the original inputs.

Finally, the data reaches the last layer (L3C0), the mouth of the river, where the decision is made. The accumulated knowledge from upstream is used to determine whether a passenger survives. This is the final outpouring of the river into the sea, where the accumulated sediment and water from upstream is deposited based on the conditions at the mouth.

This journey through the neural network's latent space reveals the model's decision-making logic. It starts with a broad, comprehensive view of the passengers, refines and integrates this information as it moves through the layers, and finally uses these distilled insights to make a prediction about each individual's fate.

---

\subsection*{Archetypal Path 6: L0C5→L1C1→L2C0→L3C1}

<details>
<summary>Path Statistics</summary>

\begin{verbatim}
Path: L0C5→L1C1→L2C0→L3C1
Count: 66 passengers (7.41\\% of dataset)
Survival Rate: 33.3\\%
Age: mean 25.9 years (±13.4)
Gender: 65.2\\% male, 34.8\\% female
Class: 22.7\\% first, 13.6\\% second, 63.6\\% third
Fare: mean £34.95 (±43.28)
\end{verbatim}
</details>

\subsubsection*{LLM Narrative}

Imagine this neural network as a detective piecing together clues from the Titanic disaster. Each cluster is a room where the detective pauses, examines the available evidence, and makes deductions.

Our journey begins in the murky depths of cluster L0C5. The detective finds a majority of male and third-class passengers and a survival rate of just 33.3\%. The next room, L1C1, reveals similar patterns, but now the detective notices age and fare as influencing factors. 

Moving on to L2C0, the detective sees the model refining its understanding even further. Now the passengers' class and fare are weighed heavily, painting a picture that is unfortunately representative of the societal biases of the time. 

Finally, in L3C1, the detective sees the culmination of the network's deductions. The survival rate remains low, confirming the model's prediction of non-survival for the majority of passengers in this path.

This journey through the layers of the network reveals a model sensitive to demographic disparities. It also raises the curtain on the importance of fairness in machine learning, making it clear that our detective, though impartial, is only as fair as the world it has learned from.

---

\subsection*{Archetypal Path 7: L0C7→L1C1→L2C0→L3C1}

<details>
<summary>Path Statistics</summary>

\begin{verbatim}
Path: L0C7→L1C1→L2C0→L3C1
Count: 63 passengers (7.07\\% of dataset)
Survival Rate: 52.4\\%
Age: mean 29.6 years (±15.8)
Gender: 65.1\\% male, 34.9\\% female
Class: 27.0\\% first, 31.7\\% second, 41.3\\% third
Fare: mean £36.70 (±67.72)
\end{verbatim}
</details>

\subsubsection*{LLM Narrative}

Imagine the entire neural network as a vast, intricate city, with each pathway representing a different journey through its streets. Our path, starting from L0C7 and ending at L3C1, is one such journey.

The journey begins at the city's outskirts, the L0C7 district. Here, the population is a mixed bag - a diverse group of individuals, without any clear distinguishing features. As they move towards the city center, the journey takes a turn, and they enter the L1C1 district. Here, the model has recognized some common patterns among the initial group and separated them accordingly.

As they continue their journey, they move into the L2C0 district. This district is slightly different - it's filled with more individuals from the third class, indicating the model's recognition of a pattern linking class and survival. 

Finally, the journey ends at the L3C1 district, where the model makes its final predictions. Here, the population is still predominantly male and third class, but despite these odds, over half have survived. This path through the city, through the model's internal reasoning, tells a story of survival against the odds, and how even in the face of societal biases, many still found a way to survive. 

This journey unravels the model's deep-seated associations, illuminating its learned patterns and biases - a crucial step towards ensuring its fair and responsible application.

---