\subsection{Cluster Labels for Heart Dataset}

Our LLM analysis revealed semantically meaningful cluster labels that reflect the model's risk stratification strategy:

\textbf{Layer 1 (Initial Risk Categorization):}
\begin{itemize}
    \item \textbf{L1C0}: ``High-Risk Older Males'' — Older, predominantly male patients with typical angina and elevated cholesterol
    \item \textbf{L1C1}: ``Lower-Risk Younger Individuals'' — Younger, mixed-sex patients with asymptomatic or non-anginal chest pain
\end{itemize}

\textbf{Layer 2 (Cardiovascular Health Refinement):}
\begin{itemize}
    \item \textbf{L2C0}: ``Low Cardiovascular Stress'' — Patients with low blood pressure and healthy cardiovascular indicators
    \item \textbf{L2C1}: ``Controlled High-Risk'' — Patients with controlled symptoms but underlying risks like high cholesterol
\end{itemize}

\textbf{Layer 3 (Abstract Risk Assessment):}
\begin{itemize}
    \item \textbf{L3C0}: ``Stress-Induced Risk'' — Patients with high blood pressure and exercise-induced angina
    \item \textbf{L3C1}: ``Moderate-Risk Active'' — Patients with moderate risk factors and higher maximum heart rates
\end{itemize}

\textbf{Output Layer (Final Classification):}
\begin{itemize}
    \item \textbf{OutputC0}: ``No Heart Disease'' — Final prediction of absence of heart disease
    \item \textbf{OutputC1}: ``Heart Disease Present'' — Final prediction of presence of heart disease
\end{itemize}