\section{Future Directions for Concept Trajectory Analysis}

Building upon our comprehensive GPT-2 case studies demonstrating CTA's effectiveness on transformer models, several promising research directions emerge:

\begin{itemize}
    \item \textbf{Scale CTA to Larger Language Models}: Extend our GPT-2 semantic analysis to larger transformer models (GPT-3/4, Claude, etc.) to understand how concept trajectories evolve with model scale and investigate whether semantic organization patterns generalize across model families.
    
    \item \textbf{Enhance Interpretability with Concept-Based Trajectory Annotations}: Integrate TCAV to align concept trajectories with human-defined concepts, making transitions more interpretable by linking them to meaningful semantic changes.
    
    \item \textbf{Attribute Features to Path Transitions}: Use Integrated Gradients (IG) to identify which input features are responsible for datapoint transitions between clusters, providing a clear, feature-level explanation of path dynamics.
    
    \item \textbf{Validate Trajectory Robustness with Topological Analysis}: Employ persistent homology to analyze the topological structure of activation spaces, ensuring that concept trajectories and fragmentation patterns are stable.
    
    \item \textbf{Extend to Attention-Based LLMs}: Use layer-specific labels to track token-level similarity-convergent trajectories in transformer models.
    
    \item \textbf{Explore Interesting Trajectories}: Use the interestingness score to prioritize trajectories for further study, especially in reinforcement learning or policy networks.
    
    \item \textbf{Interactive Visualization Tools}: Develop interactive interfaces that allow users to explore trajectories, view narratives on demand, and interactively probe the model's behavior.
    
    \item \textbf{Domain-Specific Applications}: Adapt CTA to specific domains like healthcare, finance, and natural language processing, with domain-informed metrics and explanations.
    
    \item \textbf{Enhanced Trajectory Narratives with Domain-Specific Context}: Enrich trajectory narratives by incorporating domain-specific knowledge and contextual information. This will include demographic data, domain expertise, and specialized terminologies tailored to each application domain (e.g., medical, financial, social sciences). By providing LLMs with this contextual information, narratives will become more accurate, relevant, and accessible to domain experts, facilitating better model interpretability within specific fields.
\end{itemize}

\begin{figure}[ht]
    \centering
    \includegraphics[width=0.8\textwidth]{figures/optimal_clusters.png}
    \caption{Analysis of optimal cluster counts across layers, showing how representational complexity evolves through the network.}
    \label{fig:optimal_clusters}
\end{figure}