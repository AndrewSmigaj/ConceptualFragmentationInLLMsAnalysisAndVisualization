\section{LLM-Powered Analysis for Cluster Paths}

Recent advances in large language models (LLMs) provide new opportunities for interpreting neural network behavior through the analysis of cluster paths. We introduce a systematic framework for leveraging LLMs to generate human-readable narratives and insights about the internal decision processes represented by cluster paths.

\subsection{LLM Integration Architecture}

Our framework integrates LLMs into the cluster path analysis pipeline through a modular architecture with three primary components:

\begin{enumerate}
    \item \textbf{Cluster Labeling}: LLMs analyze cluster centroids to generate meaningful semantic labels that describe the concepts each cluster might represent.
    \item \textbf{Path Narrative Generation}: LLMs create coherent narratives explaining how concepts evolve through the network as data points traverse different clusters.
    \item \textbf{Bias Audit}: LLMs analyze demographic statistics associated with paths to identify potential biases in model behavior.
\end{enumerate}

The architecture includes:

\begin{itemize}
    \item \textbf{Cache Management}: Responses are cached to enable efficient re-analysis and promote reproducibility
    \item \textbf{Prompt Optimization}: Specialized prompting techniques that improve consistency and relevance of generated content
    \item \textbf{Batch Processing}: Efficient parallel processing of multiple clusters and paths
    \item \textbf{Demography Integration}: Analysis of how cluster paths relate to demographic attributes
\end{itemize}

\subsection{Semantic Cluster Labels}

The cluster labeling process transforms abstract mathematical representations (centroids) into semantically meaningful concepts. Our experiments with the Titanic and Heart Disease datasets demonstrated that LLMs can generate consistent, meaningful labels even for similar clusters, distinguishing subtle differences in their representations.

% Include generated cluster labels
\subsection{Cluster Labels for Titanic Dataset}
\begin{itemize}
\end{itemize}
\subsection{Cluster Labels for Heart Dataset}

Our LLM analysis revealed semantically meaningful cluster labels that reflect the model's risk stratification strategy:

\textbf{Layer 1 (Initial Risk Categorization):}
\begin{itemize}
    \item \textbf{L1C0}: ``High-Risk Older Males'' — Older, predominantly male patients with typical angina and elevated cholesterol
    \item \textbf{L1C1}: ``Lower-Risk Younger Individuals'' — Younger, mixed-sex patients with asymptomatic or non-anginal chest pain
\end{itemize}

\textbf{Layer 2 (Cardiovascular Health Refinement):}
\begin{itemize}
    \item \textbf{L2C0}: ``Low Cardiovascular Stress'' — Patients with low blood pressure and healthy cardiovascular indicators
    \item \textbf{L2C1}: ``Controlled High-Risk'' — Patients with controlled symptoms but underlying risks like high cholesterol
\end{itemize}

\textbf{Layer 3 (Abstract Risk Assessment):}
\begin{itemize}
    \item \textbf{L3C0}: ``Stress-Induced Risk'' — Patients with high blood pressure and exercise-induced angina
    \item \textbf{L3C1}: ``Moderate-Risk Active'' — Patients with moderate risk factors and higher maximum heart rates
\end{itemize}

\textbf{Output Layer (Final Classification):}
\begin{itemize}
    \item \textbf{OutputC0}: ``No Heart Disease'' — Final prediction of absence of heart disease
    \item \textbf{OutputC1}: ``Heart Disease Present'' — Final prediction of presence of heart disease
\end{itemize}

\subsection{Path Narratives}

The narrative generation process explains how concepts evolve as data traverses the network. These narratives provide several interpretability advantages:

\begin{enumerate}
    \item \textbf{Contextual Integration}: Incorporating cluster labels, convergent points, fragmentation scores, and demographic data creates multi-faceted narratives.
    \item \textbf{Conceptual Evolution}: Narratives explain how concepts transform and evolve through network layers.
    \item \textbf{Decision Process Insights}: Explanations reveal potential decision-making processes that might be occurring within the model.
    \item \textbf{Demographic Awareness}: Including demographic information ensures narratives consider fairness and bias implications.
\end{enumerate}

% Include generated path narratives (These are now superseded by the _report.tex files)
% \subsection{Path Narratives for Titanic Dataset} 
% \subsection{Path Narratives for Heart Dataset}

% \subsection{Bias Audit Results} % Entire subsection commented out
% 
% The bias audit component analyzes potential demographic biases in cluster paths, creating a comprehensive analysis that:
% 
% \begin{enumerate}
%     \item \textbf{Identifies Demographic Patterns}: Reveals which demographic factors most strongly influence clustering patterns.
%     \item \textbf{Quantifies Bias}: Uses statistical measures (Jensen-Shannon divergence) to quantify deviation from baseline demographic distributions.
%     \item \textbf{Highlights Problematic Paths}: Identifies specific paths with high bias scores for further investigation.
%     \item \textbf{Provides Mitigation Strategies}: Offers concrete recommendations for addressing identified biases.
% \end{enumerate}
% 
% % Include generated bias metrics
% % No bias metrics available for titanic
% % No bias metrics available for heart

% Comprehensive GPT-4 reports
\subsection{LLM Consolidated Reports}
\section*{Holistic System Analysis Report: Titanic Dataset\n(Details for top 7 archetypes were included in the prompt)\n\n# 1. Holistic Interpretation}
- This 3-layer feedforward neural network seems to process data in stages of abstraction and convergence. The model starts with a larger set of input data and gradually refines and distills it down to a final decision.
- The high Path-Centroid Fragmentation (FC) score of 1.000 suggests that the model is highly specialized with each path being very distinct and separate from others. This could imply that the model is processing data based on very specific criteria or patterns.
- The Intra-Class Cluster Entropy (CE) of 0.862 ± 0.018 suggests that the model is making decisions with a moderate amount of uncertainty. This could indicate that the model is using a balance of exploration and exploitation to make decisions.
- The Sub-space Angle Fragmentation (SA) of 6.5° ± 3.6° shows that the model's decision boundaries are quite spread out in the input space, suggesting that the model is capturing a variety of different patterns in the data.
- The Optimal Cluster Count k* of 4.5 suggests that the model has a preference for smaller clusters, indicating that it is splitting the data into more specific and narrow categories.

\section*{2. Comparative Path Analysis}
- Archetypal paths 1, 2, and 3 all end in the same cluster (L1C0$\rightarrow$L2C1$\rightarrow$L3C0), but start from different clusters. This suggests that these paths represent different types of passengers who all ended up in the same final state.
- Archetype 4 differs from the others in that it takes a different path from the second layer onwards (L1C1$\rightarrow$L2C0$\rightarrow$L3C1). This implies that this path is associated with a different type of decision-making process, possibly related to different passenger characteristics.
- Archetype 5 is similar to archetypes 1, 2, and 3 but starts from a different cluster, suggesting that it represents a different type of passenger who also ends up in the same final state.

\section*{3. Key Decision-Making Insights}
- The fact that several paths converge to the same final cluster (L3C0) suggests that the model is making decisions based on a common set of criteria or factors.
- The variation in the initial clusters suggests that the model is taking different types of input data (e.g., age, sex, passenger class, fare) into account when making decisions.

\section*{4. System-Wide Fairness \& Bias Assessment}
- The fact that the majority of passengers in all paths are male could indicate a potential gender bias in the model's decision-making process.
- The variation in passenger class across the different paths suggests that the model is treating passengers from different classes differently. This could potentially lead to unfair outcomes for passengers in lower classes.

\section*{5. Synthesis and Story}
- The 3-layer feedforward neural network model processes Titanic dataset in a highly specialized way, focusing on specific patterns in the data to make decisions. The model takes a variety of factors into account, including age, sex, passenger class, and fare.
- Archetype 1 represents older male passengers in lower classes who paid lower fares. They generally did not survive.
- Archetype 2 represents slightly younger male passengers in higher classes who paid higher fares. They had a slightly higher survival rate.
- Archetype 3 represents younger male passengers in a mix of classes who paid moderate fares. They generally did not survive.
- Archetype 4 represents a balanced mix of male and female passengers from a range of ages and classes who paid higher fares. They had a moderate survival rate.
- Archetype 5 represents younger male passengers in a mix of classes who paid lower fares. They had a moderate survival rate.
- Overall, the model seems to favor passengers who are younger, male, in higher classes, and who paid higher fares. This suggests potential biases in the model's decision-making process. % This should contain the cleaned Titanic narratives
\section*{Holistic System Analysis Report: Heart Dataset\n(Details for top 7 archetypes were included in the prompt)\n\n# 1. Holistic Interpretation}
- The model appears to rely on a hierarchical processing strategy, where the input data is passed through successive layers of the network. At each layer, the model refines its understanding of the data, abstracting higher-level features by combining lower-level ones.
- The mean Path-Centroid Fragmentation (FC) of 1.0 indicates that the model's processing is fairly consistent and stable. This means that the model is not splitting inputs into many diverse paths, but instead tends to channel inputs through a small number of common paths.

\section*{2. Comparative Path Analysis}
- The different paths show variations in the age, sex, and cholesterol level of the inputs they process, as well as in their target distribution. For instance, Archetype 1 and Archetype 2 show a relatively balanced target distribution, while Archetype 3 and Archetype 4 have a majority of 'Present' cases.
- While all paths start from the same cluster in Layer 0, they diverge at Layer 1. This suggests that the model uses the first layer to make critical decisions about which path an input should follow.

\section*{3. Key Decision-Making Insights}
- The model seems to use age, sex, and cholesterol level as key features for decision-making. For instance, Archetype 4, which has the highest mean age and cholesterol level, also has the highest percentage of 'Present' cases.
- The model's decision-making process is not always linear or intuitive. For instance, Archetype 5, despite having a high mean age and cholesterol level, has a majority of 'Absent' cases.

\section*{4. System-Wide Fairness \& Bias Assessment}
- The model does not seem to show a clear bias towards any specific demographic group. However, the fact that different paths process inputs with different demographic characteristics raises the possibility of bias.
- If certain demographic groups are consistently routed through specific paths that have distinct fragmentation characteristics, this could suggest that the model is processing their data in a unique way. For instance, older individuals with higher cholesterol levels might consistently be routed through Archetype 4, which could be a potential source of bias.

\section*{5. Synthesis and Story}
- The model processes heart data by passing it through a series of layers, each of which refines the model's understanding of the data. The model seems to rely heavily on age, sex, and cholesterol level to make its decisions, though the relationship between these features and the output is not always straightforward.
- Archetype 1 is a well-traveled path that processes a wide range of individuals, with a slight majority of 'Absent' cases. Archetype 2, on the other hand, seems to handle individuals with slightly higher mean age and cholesterol levels, and has a more balanced distribution of 'Present' and 'Absent' cases.
- Archetype 3 and 4 are less common paths that nevertheless play a crucial role in the model's processing. Archetype 3 handles a mix of individuals but leans towards 'Present' cases, while Archetype 4 seems to specialize in older individuals with high cholesterol levels, who are mostly 'Present'.
- The moral of the story is that while the model uses a consistent and stable processing approach, its decision-making logic is complex and nuanced, relying on a combination of age, sex, and cholesterol level to make its decisions.   % This should contain the cleaned Heart narratives

\subsection{Fragmentation Metrics Overview}

To ground the LLM narratives in quantitative evidence, we compute 
three complementary fragmentation measures for every layer and for each
archetypal path:

\begin{description}
    \item[Path--Centroid Fragmentation (\textit{FC})]  Measures how
    dissimilar consecutive clusters are along a specific sample path.
    It is defined as $\mathrm{FC}=1-\overline{\mathrm{sim}}$, where
    $\overline{\mathrm{sim}}$ is the mean centroid similarity
    (cosine) between successive clusters on the path.  High values
    indicate that the representation for a data point ``jumps'' across
    concept regions between layers; low values indicate a coherent,
    incremental refinement.

    \item[Intra--Class Cluster Entropy (\textit{CE})]  For every
    layer we cluster activations and measure the Shannon entropy of
    the resulting cluster distribution \emph{within} each ground--truth
    class.  The entropy is normalised by $\log_2 k^*$ (the selected
    number of clusters) so that $\textit{CE}=1$ means class features
    are maximally dispersed, whereas $\textit{CE}=0$ means that each
    class occupies a single, compact cluster.

    \item[Sub--space Angle Fragmentation (\textit{SA})]  We compute
    the principal components for the activations of each class and
    evaluate the pair-wise principal angles between those subspaces.
    Large mean angles ($\gg 0^\circ$) imply that the network embeds
    classes in orthogonal directions—evidence of fragmentation—while
    small angles suggest a shared, low–dimensional manifold.
\end{description}

The optimal cluster count $k^*$ chosen by the silhouette criterion is
reported alongside the metrics.  Empirically we find:
\begin{itemize}
    \item Layers with \textit{CE} $\uparrow$ and \textit{SA} $\uparrow$
          correlate with high \textit{FC} in their outgoing paths,
          indicating simultaneous dispersion across samples and
          conceptual jumps along individual trajectories.
    \item Decreases in $k^*$ often coincide with lower entropy and
          angle, signalling that the network is condensing disparate
          features into fewer, more stable concepts as depth
          increases.
\end{itemize}

All three metrics are provided to the LLM as part of the prompt so
that narrative explanations can tie qualitative descriptions to
quantitative evidence (e.g., ``\emph{entropy drops sharply from layer~2
 to layer~3, indicating that the network consolidates passenger class
 information}'').

\begin{figure}[t]
    \centering
    \includegraphics[width=0.32\textwidth]{figures/optimal_clusters.png}
    \includegraphics[width=0.32\textwidth]{figures/cluster_entropy.png}
    \includegraphics[width=0.32\textwidth]{figures/subspace_angle.png}
    \caption{Layer-wise quantitative fragmentation metrics.
    Left: optimal $k^*$ per layer;
    middle: normalised intra-class cluster entropy (CE);
    right: mean sub-space angle (SA).}
    \label{fig:fragmentation_metrics}
\end{figure}

\begin{table}[h!]
\centering
\caption{Layer-wise fragmentation metrics for the Titanic dataset model.}
\label{tab:fragmentation_metrics_titanic}
\begin{tabular}{lcccc}
\toprule
Layer & $k^*$ & CE & SA ($^\circ$) & FC (path mean) \\
\midrule
Layer 1 & 10 & 0.818 & 39.6 & 0.432 \\
Layer 2 &  2 & 0.784 & 33.5 & 0.432 \\
Layer 3 &  2 & 0.779 & 25.7 & 0.432 \\
Output  &  2 & 0.761 & 12.9 & 0.432 \\
\bottomrule
\end{tabular}
\end{table}

\begin{table}[h!]
\centering
\caption{Layer-wise fragmentation metrics for the Heart dataset model.}
\label{tab:fragmentation_metrics_heart}
\begin{tabular}{lcccc}
\toprule
Layer & $k^*$ & CE & SA ($^\circ$) & FC (path mean) \\
\midrule
Layer 1 & 2 & 0.722 & 16.3 & 0.096 \\
Layer 2 & 2 & 0.713 & 11.5 & 0.096 \\
Layer 3 & 2 & 0.711 &  7.8 & 0.096 \\
Output  & 2 & 0.702 &  3.1 & 0.096 \\
\bottomrule
\end{tabular}
\end{table}

\subsection{Advantages and Limitations}

\textbf{Advantages}:
\begin{enumerate}
    \item \textbf{Interpretable Insights}: Converts complex mathematical patterns into human-readable explanations.
    \item \textbf{Multi-level Analysis}: Provides insights at cluster, path, and system-wide levels.
    \item \textbf{Bias Detection}: Proactively identifies potential fairness concerns in model behavior.
    \item \textbf{Integration with Metrics}: Combines qualitative narratives with quantitative fragmentation and similarity metrics.
\end{enumerate}

\textbf{Limitations}:
\begin{enumerate}
    \item \textbf{Potential for Overinterpretation}: LLMs might ascribe meaning to patterns that are artifacts of the clustering process.
    \item \textbf{Domain Knowledge Gaps}: Analysis quality depends on the LLM's understanding of the specific domain.
    \item \textbf{Computational Cost}: Generating narratives for many paths can be resource-intensive.
    \item \textbf{Validation Challenges}: Verifying the accuracy of generated narratives requires domain expertise.
\end{enumerate}

% The following placeholder prose was part of an earlier draft and is now
% superseded by automatically generated cluster labels, narratives and bias
% tables inserted via \input.  To avoid contradictory text we comment it out.
\iffalse
Our experiments show that these narratives can effectively translate complex mathematical relationships into intuitive explanations that capture the essence of the model's internal behavior.

### 6.4 Bias Auditing Through LLMs
... (placeholder content removed) ...
\fi