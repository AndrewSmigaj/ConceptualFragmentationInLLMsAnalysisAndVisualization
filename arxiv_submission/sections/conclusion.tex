\section{Conclusion}

Concept Trajectory Analysis (CTA) represents a significant advance in neural network interpretability by providing a principled framework for understanding how information flows and transforms through network layers. By combining rigorous mathematical foundations with accessible visualization tools, CTA bridges the gap between technical analysis and practical understanding of neural network behavior.

The methodology's core contribution lies in its ability to track concept evolution dynamically. Rather than analyzing individual layers in isolation, CTA reveals how representations transform across the entire network, uncovering organizational principles that static analyses miss. The integration of cross-layer metrics—centroid similarity ($\rho^c$), membership overlap ($J$), and trajectory fragmentation (F)—provides quantitative rigor, while LLM-powered narrative generation ensures human interpretability.

Our case studies demonstrate CTA's versatility across domains. In language models, we discovered unexpected organizational principles that challenge assumptions about semantic processing. In medical AI, we showed how CTA can enhance trust and safety by revealing decision pathways and identifying potential biases. These applications validate CTA's practical value beyond theoretical interest.

The open-source Concept MRI tool democratizes access to these interpretability methods. By providing interactive visualizations and automated analysis pipelines, we enable researchers and practitioners to apply CTA to their own models and domains. The tool's modular architecture supports extension to new network types and analysis techniques.

As AI systems become more prevalent in critical applications, the need for interpretability grows urgent. CTA offers a path forward—not just for understanding current models, but for designing more interpretable architectures from the ground up. By revealing how neural networks organize information, we move closer to AI systems that are both powerful and trustworthy.

\subsection{Limitations and Failure Modes}

While CTA provides valuable insights into neural network organization, several limitations and failure modes warrant discussion:

\subsubsection{Technical Limitations}

\begin{itemize}
    \item \textbf{Clustering instability}: When activation spaces lack clear structure, clustering results may vary significantly across runs. Low silhouette scores or high variance in cluster assignments indicate unreliable trajectories.
    \item \textbf{Scalability challenges}: Very deep networks (100+ layers) pose computational and interpretability challenges. Tracking trajectories through many layers can obscure rather than clarify patterns.
    \item \textbf{High-dimensional curse}: In extremely high-dimensional activation spaces, distance metrics become less meaningful, potentially leading to arbitrary cluster assignments.
\end{itemize}

\subsubsection{Interpretation Risks}

\begin{itemize}
    \item \textbf{Spurious patterns}: Random fluctuations might appear as meaningful paths, especially with small sample sizes. Statistical validation (as we demonstrated with $\chi^2$ tests) is essential.
    \item \textbf{LLM hallucination}: Generated narratives may sound plausible while misrepresenting actual patterns. Cross-validation with quantitative metrics is crucial.
    \item \textbf{Correlation vs. causation}: CTA reveals organizational patterns but cannot establish causal relationships. Interventional studies are needed to verify causal claims.
\end{itemize}

\subsubsection{Application Boundaries}

\begin{itemize}
    \item \textbf{Architecture dependence}: CTA works best with feedforward architectures. Recurrent or highly branched architectures may require adaptation.
    \item \textbf{Domain transfer}: Patterns discovered in one domain (e.g., language) may not transfer to others (e.g., vision) without careful validation.
    \item \textbf{Training dynamics}: CTA analyzes trained models. Understanding how these patterns emerge during training requires additional analysis.
\end{itemize}

Despite these limitations, CTA's mathematical grounding and statistical validation demonstrate its potential as an interpretability tool. The patterns observed in GPT-2 and medical AI applications warrant further investigation. Responsible use involves acknowledging these limitations, validating findings through multiple approaches, and maintaining appropriate skepticism about generated narratives.
