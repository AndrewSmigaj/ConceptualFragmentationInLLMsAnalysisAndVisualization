\documentclass[11pt,a4paper]{article}
\usepackage[utf8]{inputenc}
\usepackage[T1]{fontenc}
\usepackage{amsmath}
\usepackage{amsfonts}
\usepackage{amssymb}
\usepackage{graphicx}
\usepackage{hyperref}
\usepackage{xcolor}
\usepackage{booktabs}
\usepackage{algorithm}
\usepackage{algorithmic}
\usepackage{listings}
\usepackage{caption}
\usepackage{subcaption}
\usepackage{multirow}
\usepackage{bm}
\usepackage{natbib}

% Define colors for hyperlinks
\hypersetup{
    colorlinks=true,
    linkcolor=blue,
    filecolor=magenta,
    urlcolor=cyan,
}

% Code listing style
\lstset{
    basicstyle=\ttfamily\small,
    commentstyle=\color{gray},
    keywordstyle=\color{blue},
    stringstyle=\color{green!50!black},
    numbers=left,
    numberstyle=\tiny\color{gray},
    numbersep=5pt,
    breaklines=true,
    showstringspaces=false,
    frame=single,
    captionpos=b
}

% Title information
\title{Foundations of Archetypal Path Analysis: \\
Toward a Principled Geometry for Cluster-Based Interpretability with LLM-Powered Narrative Explanation}
\author{Andrew Smigaj$^1$, Claude Anthropic$^2$, Grok xAI$^3$ \\
$^1$University of Technical Sciences, $^2$Anthropic Research, $^3$xAI Corp}
\date{July 2025}

\begin{document}

\maketitle

% Abstract
\begin{abstract}
We present Concept Trajectory Analysis (CTA), an interpretability method that tracks how concepts move through clustered activation spaces in neural networks. Applying CTA to GPT-2, we analyzed 1,228 single-token words across 8 semantic categories with balanced grammatical representation (33.1\% verbs). We observed that 48.5\% (95\% CI: 45.7\%–51.2\%) of words converge to a dominant pathway we term the ``entity superhighway,'' where nouns cluster together regardless of whether they represent animals, objects, or abstract concepts. This pattern, validated by highly significant grammatical clustering ($\chi^2 = 95.90$, $p < 0.0001$), demonstrates that GPT-2 uses grammatical function as its primary organizing principle while maintaining semantic distinctions within these grammatical pathways. 

CTA combines mathematical rigor with human interpretability: we formalize geometric conditions for valid clustering, introduce windowed analysis (Early/Middle/Late) that reveals phase transitions in neural processing, and employ unique cluster labeling (L\{layer\}\_C\{cluster\}) to track concept evolution precisely. Our framework quantifies path diversity through trajectory fragmentation metrics while leveraging LLMs to generate interpretable cluster labels and insights.

Beyond transformer analysis, we applied CTA to traditional ML tasks. In heart disease diagnosis, trajectory analysis shows how models process patient data through distinct pathways. Different patient profiles—distinguished by age, cholesterol, and clinical markers—tend to follow different neural pathways, with path fragmentation potentially indicating diagnostic uncertainty. This application demonstrates CTA's utility for understanding model decision-making in healthcare contexts. By providing visibility into neural network organization, CTA offers a tool for model interpretation that could be useful for debugging and analysis.

\textbf{Research Question} --- One-Sentence Focus: Under what geometric and statistical conditions do layerwise activation clusters form stable, semantically meaningful concept trajectories that can be used for faithful model interpretation?
\end{abstract}

% Introduction
\section{Introduction}

Interpretability research often struggles to balance rigor and accessibility, oscillating between visually compelling but loosely grounded ``quasi-explanations'' and mathematically sound but opaque theoretical analyses. Concept Trajectory Analysis (CTA) aims to bridge this gap by clustering datapoint activations at each layer of a trained network and tracing their transitions through activation space. This approach provides both mathematical precision and intuitive understanding of how neural networks process information, but its utility depends on addressing several foundational questions:

\begin{itemize}
    \item What makes activation geometry suitable for clustering?
    \item Under what transformations are cluster paths stable?
    \item Do cluster paths reflect genuine semantic or decision-relevant structure?
    \item How can we translate mathematical patterns into domain-meaningful explanations?
\end{itemize}

The challenge of translating quantitative patterns into qualitative understanding is particularly acute. While metrics like silhouette scores or mutual information can validate clustering quality, they offer limited insight into the semantic meaning of identified patterns. Our work introduces a novel approach to this problem by leveraging large language models (LLMs) to generate human-readable narratives that explain the conceptual significance of activation patterns.

We demonstrate this methodology through groundbreaking case studies analyzing GPT-2's internal representations. Our semantic pivot analysis reveals how transformer models process contradictory information. Most remarkably, our semantic subtypes study—examining 566 validated words across 8 semantic categories—discovered that GPT-2 organizes language by grammatical function rather than semantic meaning. Through our novel ``Concept MRI'' visualization, we found that 72.8\% of all words converge to a single ``noun superhighway,'' with animals and objects processed identically despite semantic differences. This counterintuitive finding challenges fundamental assumptions about how neural language models organize information.

\textbf{Contributions}:
\begin{itemize}
    \item Formalize activation-space geometry for datapoint clustering with layer-specific labels and mathematical validation criteria
    \item Introduce ETS-based clustering for dimension-wise explainability with verbalizably transparent membership conditions
    \item Develop cross-layer metrics (centroid similarity, membership overlap, fragmentation scores) to quantify concept evolution
    \item Propose a reproducible framework for path stability assessment using statistical robustness measures (ARI, MI, null models)
    \item Implement LLM-powered analysis to generate human-readable explanations of cluster paths
    \item \textbf{Discover grammatical organization in GPT-2}: First empirical evidence that transformers organize by syntax not semantics
    \item \textbf{Introduce ``Concept MRI'' visualization}: Novel technique for tracking concept flow through neural networks
    \item \textbf{Reveal massive convergence patterns}: 72.8\% of words converge to unified grammatical pipelines
    \item \textbf{Identify phase transitions}: Stability analysis reveals critical reorganization points in neural processing
    \item Demonstrate bias detection capabilities through demographic analysis of paths
    \item Provide an open-source implementation blueprint for application across diverse domains
\end{itemize}

% Background and Motivation
\section{Background and Motivation}

\subsection{Concept Trajectory Analysis Recap}

Concept Trajectory Analysis (CTA) clusters datapoint activations in each layer's activation space (e.g., using k-means or DBSCAN), assigning layer-specific cluster IDs, denoted $(L_lC_k)$, where $(l)$ is the layer index and $(k)$ is the cluster index (e.g., L1C0 for cluster 0 in layer 1). Transitions between clusters are tracked across layers, forming trajectories $\pi_i = [c_i^1, c_i^2, \dots, c_i^L]$, interpreted as latent semantic concept evolution. In feedforward networks, trajectories are strictly unidirectional, and clusters with different layer-specific IDs (e.g., L1C0 and L3C0) are not assumed to be related unless validated by geometric similarity (e.g., centroid cosine similarity). Large language models (LLMs) can narrate these trajectories to provide interpretable insights.

\subsection{Critique from Foundational Viewpoints}

Activation spaces are emergent, high-dimensional representations whose coordinate systems may not map to semantically meaningful axes. Without grounding, Euclidean distances can be misleading, and clustering is sensitive to initialization and density assumptions. Prior work echoes this concern: \citet{ribeiro2016} and \citet{lundberg2017} show that LIME and SHAP trade exactness for intuition; \citet{dasgupta2020} advocate explainable clustering by rule-based thresholds rather than arbitrary distances.

% Mathematical Foundation
\section{Mathematical Foundation of Layerwise Activation Geometry}

\subsection{What Is Being Clustered?}

Let $A^l \in \mathbb{R}^{n \times d_l}$ denote the matrix of activations at layer $(l)$, where each row $\mathbf{a}_i^l$ is a datapoint's activation vector. Once the model is trained, $A^l$ provides a static representation space per layer.

\subsection{Metric Selection and Validity}

We cluster $A^l$ into $k_l$ clusters, assigning unique layer-specific labels L$l$\_C0, L$l$\_C1, \dots, L$l$\_C$\{k_l-1\}$. This unique labeling scheme (e.g., L4\_C1 for layer 4, cluster 1) prevents cross-layer confusion and enables precise tracking of concept evolution. A datapoint's path is a sequence $\pi_i = [c_i^1, c_i^2, \dots, c_i^L]$, where $c_i^l$ is the cluster assignment at layer $l$ in the format L$l$\_C$k$. 

We determine optimal $k_l$ using the Gap statistic, which compares within-cluster dispersion to that expected under a null reference distribution. For layer $l$, we compute:
$$\text{Gap}(k) = \mathbb{E}[\log(W_k^*)] - \log(W_k)$$
where $W_k$ is the within-cluster sum of squares and $W_k^*$ is its expectation under the null.

We examine Euclidean, cosine, and Mahalanobis metrics. In high-dimensional spaces, Euclidean norms lose contrast; cosine and L1 often behave better. PCA or normalization can stabilize comparisons. In feedforward networks, paths are unidirectional, and apparent convergence (e.g., [L1\_C0 $\rightarrow$ L2\_C2 $\rightarrow$ L3\_C0]) is validated by computing cosine or Euclidean similarity between cluster centroids across layers, ensuring that any perceived similarity reflects geometric proximity in activation space rather than shared labels.


\subsection{Clustering Approaches}

We primarily use k-means clustering with the Gap statistic for determining optimal cluster counts.

\subsection{Within-Cluster Semantic Structure}

While our primary analysis focuses on cluster-level trajectories, the position of datapoints within clusters may carry semantic meaning. Following principles of distributional semantics, nearby points in activation space often share semantic properties—even within the same cluster. For instance, within the dominant entity pathway (L4\_C1), "cat" and "dog" may occupy closer positions than "cat" and "democracy," despite all being nouns. This suggests potential hierarchical organization where coarse clusters capture grammatical categories while fine-grained positions encode semantic relationships.

Future work could explore micro-clustering within major pathways to investigate these potential semantic substructures. Techniques like hierarchical clustering or local neighborhood analysis might reveal how dominant pathways subdivide into semantic regions while maintaining overall grammatical coherence.

\subsection{Windowed Trajectory Analysis}

To capture phase transitions in neural processing, we introduce windowed analysis that segments the network into functional regions:
\begin{itemize}
    \item \textbf{Early Window} (layers 0-3): Initial feature extraction and semantic differentiation
    \item \textbf{Middle Window} (layers 4-7): Conceptual reorganization and consolidation
    \item \textbf{Late Window} (layers 8-11): Final representation and task-specific processing
\end{itemize}

For each window $w$, we compute stability metrics:
$$S_w = \frac{1}{|P_w|} \sum_{p \in P_w} \frac{|\text{mode}(p)|}{|p|}$$
where $P_w$ is the set of path segments in window $w$ and $\text{mode}(p)$ is the most frequent cluster transition. Changes in stability patterns across windows can indicate phase transitions in the network's organizational principles.


\subsection{Quantitative Metrics for Concept Evolution}

To ground our analysis in quantitative evidence, we employ four complementary metrics that capture different aspects of concept evolution through neural networks:

\subsubsection{Trajectory Fragmentation (F)}
Measures path diversity for a semantic category:
$$F = 1 - \frac{\text{count of most common path}}{\text{total paths in category}}$$
High fragmentation indicates diverse processing strategies within a category. In our experiments, this metric helps quantify convergence patterns—for instance, the GPT-2 analysis shows fragmentation varying from 0.796 (early) to 0.499 (middle) to 0.669 (late), suggesting complex dynamics in the organization of the balanced dataset.

\subsubsection{Path-Centroid Fragmentation (FC)}
Measures how dissimilar consecutive clusters are along a specific sample path:
$$FC = 1 - \overline{\text{sim}}$$
where $\overline{\text{sim}}$ is the mean centroid similarity (cosine) between successive clusters on the path. High values indicate that representations "jump" across concept regions between layers; low values indicate coherent, incremental refinement. The heart disease model shows remarkably low FC=0.096, indicating smooth transitions.

\subsubsection{Intra-Class Cluster Entropy (CE)}
For every layer, we cluster activations and measure the Shannon entropy of the resulting cluster distribution within each ground-truth class:
$$CE = \frac{H(C|Y)}{\log_2 k^*}$$
where $H(C|Y)$ is the conditional entropy of clusters given class labels, normalized by $\log_2 k^*$ (the selected number of clusters). CE=1 means class features are maximally dispersed across clusters, while CE=0 means each class occupies a single, compact cluster.

\subsubsection{Sub-space Angle Fragmentation (SA)}
We compute the principal components for the activations of each class and evaluate the pair-wise principal angles between those subspaces. Large mean angles ($\gg 0°$) imply that the network embeds classes in orthogonal directions—evidence of fragmentation—while small angles suggest a shared, low-dimensional manifold. In GPT-2, we observe SA collapsing from 45-60° (semantic separation) to 5-10° (grammatical convergence).

\subsection{Applying the Framework: From Theory to Practice}

These metrics work in concert to reveal different aspects of neural organization. In Section \ref{sec:gpt2_case_study}, we apply them to uncover GPT-2's grammatical organization, where decreasing SA and CE values quantify the convergence from semantic to syntactic processing. In Section \ref{sec:heart_case_study}, consistently low FC values validate that medical diagnosis models maintain coherent patient representations throughout processing. The windowed analysis framework proves particularly powerful for identifying phase transitions—critical reorganization points where networks shift their organizational principles, as evidenced by stability metric drops in GPT-2's middle layers.



% Statistical Robustness
\section{Statistical Robustness of Cluster Structures}

This section extends the mathematical foundation (Section 4) with additional statistical analyses that validate the robustness of discovered cluster structures and paths.

\subsection{Path Reproducibility Across Seeds}

To assess structural stability, we define dominant archetypal paths as frequent cluster sequences across datapoints. We compute Jaccard overlap and recurrence frequency of these paths across different random seeds or bootstrapped model runs. High path recurrence suggests the presence of model-internal decision logic rather than sampling artifacts. 

For clusters L$l$\_C$k$ and L$m$\_C$j$, let $S_k^l = \{i \mid \text{datapoint } i \text{ in cluster L}l\text{\_C}k\}$. We compute Jaccard similarity, $J(S_k^l, S_j^m) = \frac{|S_k^l \cap S_j^m|}{|S_k^l \cup S_j^m|}$, to measure datapoint retention across layers. High overlap between clusters -- with high centroid similarity suggests stable group trajectories. 

We also compute the frequency of similarity-convergent paths by aggregating transitions where the final cluster resembles an earlier one, e.g., $[\text{L1\_C}k \rightarrow \text{L2\_C}j \rightarrow \text{L3\_C}m]$ where $\text{cos}(\mathbf{c}_k^1, \mathbf{c}_m^3) > 0.9$. Density is calculated as $D = \sum_{\text{similarity-convergent paths}} T^1_{kj} T^2_{jm} \cdot \mathbb{1}[\text{cos}(\mathbf{c}_k^1, \mathbf{c}_m^3) > \theta]$, where $T^l_{kj}$ is the transition count from L$l$\_C$k$ to L$(l+1)$\_C$j$. High density suggests latent funnels where datapoints converge to similar activation spaces.

\subsection{Trajectory Coherence}

For a datapoint $(i)$ with path $\pi_i = [c_i^1, c_i^2, \dots, c_i^L]$, we compute the trajectory coherence score using subspace angles between consecutive centroid transitions: 
\begin{align}
TC_i = \frac{1}{L-2} \sum_{t=2}^{L-1} \arccos\left(\frac{(\mathbf{c}_{c_i^{t+1}}^{t+1} - \mathbf{c}_{c_i^t}^t) \cdot (\mathbf{c}_{c_i^t}^t - \mathbf{c}_{c_i^{t-1}}^{t-1})}{\|\mathbf{c}_{c_i^{t+1}}^{t+1} - \mathbf{c}_{c_i^t}^t\| \|\mathbf{c}_{c_i^t}^t - \mathbf{c}_{c_i^{t-1}}^{t-1}\|}\right)
\end{align}

Low $TC_i$ indicates coherent trajectories, especially in similarity-convergent paths. Note that this trajectory coherence metric differs from the trajectory fragmentation ($F$) defined in Section 2, which measures path diversity at the category level.



% Experimental Design
\section{Experimental Design}

Our experiments validate CTA across diverse domains, from medical AI to language understanding:

\subsection{Datasets and Models}

\begin{itemize}
    \item \textbf{Heart Disease Diagnosis}: UCI Heart Disease dataset (303 patients, 13 clinical features) with 3-layer MLP, demonstrating medical AI interpretability
    \item \textbf{GPT-2 Semantic Subtypes}: 1,228 validated single-token words across 8 semantic categories, analyzed through GPT-2's 12 layers (embedding layer + 11 transformer blocks, 117M parameters)\footnote{We follow standard convention in numbering GPT-2's layers: Layer 0 is the embedding layer, and Layers 1-11 correspond to transformer blocks 1-11. The 12th transformer block was not analyzed in this study.}
    \item \textbf{GPT-2 Semantic Pivot}: 202 sentences with contradictory information, tracking semantic processing
\end{itemize}

\subsection{Unified CTA Methodology}

We employ a three-phase approach:

\begin{enumerate}
    \item \textbf{Optimal Clustering}: Gap statistic determines layer-specific cluster counts (e.g., k=4 for GPT-2 layer 0, k=2 for layers 1-11)
    \item \textbf{Unique Labeling}: Every cluster receives globally unique ID (L\{layer\}\_C\{cluster\}) preventing cross-layer confusion
    \item \textbf{Windowed Analysis}: Temporal segmentation into Early/Middle/Late windows reveals phase transitions
\end{enumerate}

\subsection{Metrics and Validation}

\begin{itemize}
    \item \textbf{Cross-layer Metrics}: Centroid similarity ($\rho^c$), membership overlap ($J$), trajectory fragmentation ($F$)
    \item \textbf{Stability Analysis}: Window-based stability scores revealing reorganization points
    \item \textbf{Path Statistics}: Convergence ratios, diversity indices, archetype identification
    \item \textbf{Clinical Validation}: For heart disease, correlation of fragmentation with diagnostic uncertainty
\end{itemize}

\subsection{Visualization Suite}

\begin{itemize}
    \item \textbf{Concept MRI Tool}: Software implementing CTA with interactive Sankey diagrams showing complete concept flow
    \item \textbf{Clinical Dashboards}: Patient archetype visualization for medical interpretability
    \item \textbf{Interactive Exploration}: Web-based interfaces for real-time analysis
\end{itemize}




% LLM-Powered Analysis
\section{LLM-Powered Analysis for Cluster Paths}

Recent advances in large language models (LLMs) provide new opportunities for interpreting neural network behavior through the analysis of cluster paths. We introduce a systematic framework for leveraging LLMs to generate human-readable narratives and insights about the internal decision processes represented by cluster paths.

\subsection{LLM Integration Architecture}

Our framework integrates LLMs into the cluster path analysis pipeline through a modular architecture with three primary components:

\begin{enumerate}
    \item \textbf{Cluster Labeling}: LLMs analyze cluster centroids to generate meaningful semantic labels that describe the concepts each cluster might represent.
    \item \textbf{Path Narrative Generation}: LLMs create coherent narratives explaining how concepts evolve through the network as data points traverse different clusters.
    \item \textbf{Bias Audit}: LLMs analyze demographic statistics associated with paths to identify potential biases in model behavior.
\end{enumerate}

The architecture includes:

\begin{itemize}
    \item \textbf{Cache Management}: Responses are cached to enable efficient re-analysis and promote reproducibility
    \item \textbf{Prompt Optimization}: Specialized prompting techniques that improve consistency and relevance of generated content
    \item \textbf{Batch Processing}: Efficient parallel processing of multiple clusters and paths
    \item \textbf{Demography Integration}: Analysis of how cluster paths relate to demographic attributes
\end{itemize}

\subsection{Semantic Cluster Labels}

The cluster labeling process transforms abstract mathematical representations (centroids) into semantically meaningful concepts. LLMs analyze cluster properties—including centroid values, dominant features, and datapoint characteristics—to generate interpretable labels. For instance, in medical applications, clusters might be labeled as "High-Risk Elderly" or "Low Cardiovascular Stress" based on their statistical properties. This automated labeling provides immediate interpretability while maintaining consistency across analyses.

\subsection{Path Narratives}

The narrative generation process explains how concepts evolve as data traverses the network. These narratives provide several interpretability advantages:

\begin{enumerate}
    \item \textbf{Contextual Integration}: Incorporating cluster labels, convergent points, fragmentation scores, and demographic data creates multi-faceted narratives.
    \item \textbf{Conceptual Evolution}: Narratives explain how concepts transform and evolve through network layers.
    \item \textbf{Decision Process Insights}: Explanations reveal potential decision-making processes that might be occurring within the model.
    \item \textbf{Demographic Awareness}: Including demographic information ensures narratives consider fairness and bias implications.
\end{enumerate}

% Include generated path narratives (These are now superseded by the _report.tex files)
% \subsection{Path Narratives for Titanic Dataset} 
% \subsection{Path Narratives for Heart Dataset}

% \subsection{Bias Audit Results} % Entire subsection commented out
% 
% The bias audit component analyzes potential demographic biases in cluster paths, creating a comprehensive analysis that:
% 
% \begin{enumerate}
%     \item \textbf{Identifies Demographic Patterns}: Reveals which demographic factors most strongly influence clustering patterns.
%     \item \textbf{Quantifies Bias}: Uses statistical measures (Jensen-Shannon divergence) to quantify deviation from baseline demographic distributions.
%     \item \textbf{Highlights Problematic Paths}: Identifies specific paths with high bias scores for further investigation.
%     \item \textbf{Provides Mitigation Strategies}: Offers concrete recommendations for addressing identified biases.
% \end{enumerate}
% 
% % Include generated bias metrics
% % No bias metrics available for titanic
% % No bias metrics available for heart

% Note: Specific LLM-generated reports for case studies are included in their respective sections

\subsection{Integrating Metrics with Narratives}

The quantitative metrics defined in Section 3.6 (F, FC, CE, SA) are provided to the LLM as part of the prompt, enabling narrative explanations that tie qualitative descriptions to quantitative evidence. For example, the LLM can explain that "entropy drops sharply from layer 2 to layer 3, indicating that the network consolidates risk factors" or "the decreasing sub-space angles reveal progressive alignment between disease and healthy patient representations."


\begin{table}[h!]
\centering
\caption{Example layer-wise fragmentation metrics showing how different metrics capture complementary aspects of concept evolution.}
\label{tab:fragmentation_metrics_example}
\begin{tabular}{lcccc}
\toprule
Layer & $k^*$ & CE & SA ($^\circ$) & FC (path mean) \\
\midrule
Layer 1 & 2 & 0.722 & 16.3 & 0.096 \\
Layer 2 & 2 & 0.713 & 11.5 & 0.096 \\
Layer 3 & 2 & 0.711 &  7.8 & 0.096 \\
Output  & 2 & 0.702 &  3.1 & 0.096 \\
\bottomrule
\end{tabular}
\end{table}\footnote{In this example from a shallow network, the consistent FC value of 0.096 indicates stable cluster representations throughout. Low fragmentation coefficients suggest smooth concept evolution, with cluster centroids maintaining high similarity (approximately 90.4\% cosine similarity) between consecutive layers.}

\subsection{Advantages and Limitations}

\textbf{Advantages}:
\begin{enumerate}
    \item \textbf{Interpretable Insights}: Converts complex mathematical patterns into human-readable explanations.
    \item \textbf{Multi-level Analysis}: Provides insights at cluster, path, and system-wide levels.
    \item \textbf{Bias Detection}: Proactively identifies potential fairness concerns in model behavior.
    \item \textbf{Integration with Metrics}: Combines qualitative narratives with quantitative fragmentation and similarity metrics.
\end{enumerate}

\textbf{Limitations}:
\begin{enumerate}
    \item \textbf{Potential for Overinterpretation}: LLMs might ascribe meaning to patterns that are artifacts of the clustering process.
    \item \textbf{Domain Knowledge Gaps}: Analysis quality depends on the LLM's understanding of the specific domain.
    \item \textbf{Computational Cost}: Generating narratives for many paths can be resource-intensive.
    \item \textbf{Validation Challenges}: Verifying the accuracy of generated narratives requires domain expertise.
\end{enumerate}

% The following placeholder prose was part of an earlier draft and is now
% superseded by automatically generated cluster labels, narratives and bias
% tables inserted via \input.  To avoid contradictory text we comment it out.
\iffalse
Our experiments show that these narratives can effectively translate complex mathematical relationships into intuitive explanations that capture the essence of the model's internal behavior.

### 6.4 Bias Auditing Through LLMs
... (placeholder content removed) ...
\fi

% Use Cases for LLMs
\section{Use Cases for LLMs}

\begin{itemize}
    \item \textbf{Prompt Strategy Evaluation}: Compare similarity-convergent path density and fragmentation scores (see Section 3.6) across prompt framings (e.g., Socratic vs. assertive) to reveal shifts in internal decision consistency.
    \item \textbf{Layerwise Ambiguity Detection}: Identify prompt-token pairs with divergent latent paths across LLM layers, highlighting instability or multiple plausible completions.
    \item \textbf{Subgroup Drift Analysis}: Track membership overlap for datapoint groups (e.g., positive vs. negative sentiment) across layers, using centroid similarity to identify convergence.
    \item \textbf{Behavioral Explanation}: Generate LLM-authored natural language summaries for archetypal paths, e.g., ``Datapoints in L1C0, characterized by [feature], transition to L3C2, which is geometrically similar (cosine similarity 0.92), indicating [semantic consistency].''
    \item \textbf{Failure Mode Discovery}: Flag high-fragmentation paths as potential errors, e.g., misclassifications or hallucinations.
\end{itemize}

\subsection{Example Use Cases}

\begin{itemize}
    \item \textbf{Prompt Engineering}: Compare paths for prompts like ``Tell me a story'' vs. ``Write a creative tale'' to see how wording affects internal flow. The former might follow a more fragmented path, indicating uncertainty, while the latter converges quickly to a ``creative'' cluster.
    \item \textbf{Bias Detection}: Analyze paths for inputs with gender pronouns (e.g., ``he'' vs. ``she'' in professional contexts) to detect divergent behavior. A path diverging at layer 2 for ``she'' might indicate biased feature weighting.
    \item \textbf{Error Analysis}: Study paths for misclassified inputs to pinpoint failure points. A misclassified datapoint might exhibit a highly fragmented path, suggesting internal confusion.
\end{itemize}

\begin{figure}[ht]
    \centering
    \includegraphics[width=0.8\textwidth]{figures/trajectory_by_endpoint_cluster.png}
    \caption{Visualization of paths colored by their endpoint clusters, revealing how datapoints that end up in the same final representation may take different routes through the network's internal spaces.}
    \label{fig:trajectory_endpoint}
\end{figure}

% Reproducibility and Open Science
\section{Reproducibility and Open Science}

\begin{itemize}
    \item Code and configs released under MIT license at \href{https://github.com/ConceptualFragmentationInLLMsAnalysisAndVisualization}{GitHub repository}
    \item Seed lists and hyperparameters logged in JSON format
    \item Dockerfile ensures environment parity across research teams
    \item Negative results and failed variants documented in appendices
    \item LLM prompts and responses cached for reproducibility
\end{itemize}

\paragraph{Key pipeline steps (pseudocode).}

\begin{verbatim}
# 1. Train baseline model and cache activations
python train_baseline.py --dataset <dataset> --seed <seed>

# 2. Compute cluster paths and metrics
python concept_fragmentation/analysis/cluster_paths.py \
       --dataset <dataset> --seed <seed> --compute_similarity

# 3. Generate cluster labels and LLM narratives
python llm_path_analysis.py --dataset <dataset> --seed <seed>

# 4. Build LaTeX fragments & figures
python tools/build_paper_tables.py
python generate_paper_figures.py --dataset <dataset>
\end{verbatim}

Full, runnable code is available in the public repository; all prompts and
LLM responses are cached for deterministic builds.

\subsection{LLM Prompts for Cluster Interpretation}

To ensure reproducibility of our LLM-powered analysis, we document the key prompts used for cluster interpretation and path analysis:

\paragraph{Cluster Labeling Prompt:}
\begin{verbatim}
You are analyzing clusters from a neural network. 
For cluster L{layer}_C{cluster} containing these words:
{sample_words}

Category distribution: {category_counts}
Cluster size: {size} words

Provide a concise, interpretable label that captures 
the semantic or grammatical essence of this cluster.
\end{verbatim}

\paragraph{Path Narrative Prompt:}
\begin{verbatim}
Analyze this concept trajectory through GPT-2:
Path: {path}
Window: {window_name}
Grammatical distribution: {grammatical_counts}

Explain how concepts evolve through these clusters,
focusing on the transformation from semantic to 
grammatical organization.
\end{verbatim}

\paragraph{Bias Analysis Prompt:}
\begin{verbatim}
Analyze potential biases in these neural pathways:
Path: {path}
Demographics: {demographic_stats}
Outcome distribution: {outcomes}

Identify any concerning patterns or biases in how
different demographic groups are processed.
\end{verbatim}

Interactive demos and full code implementation are available on our project repository.

% Societal Impact
\section{Societal Impact}

CTA's narratives can demystify opaque models but risk misuse if interpreted uncritically. By using layer-specific cluster labels, CTA avoids misleading implications of cyclic behavior in feedforward networks, enhancing transparency. However, similarity-convergent paths risk overinterpretation as causal relationships. We mitigate this by validating convergence with centroid similarity and null-model baselines, ensuring narratives align with IEEE and EU AI guidelines.

\subsection{Limitations and Risks}

\begin{itemize}
    \item \textbf{Overinterpretation}: CTA narratives reflect patterns in activation spaces, not causal relationships. Users should avoid inferring definitive explanations from paths alone.
    \item \textbf{Scalability}: Applying CTA to massive models (e.g., LLMs with billions of parameters) may be computationally intensive. We recommend sampling datapoints or using approximate clustering methods.
    \item \textbf{Validation}: Narratives should be cross-checked with domain expertise and other interpretability methods to ensure accuracy. To mitigate these risks, LLM-generated narratives include disclaimers, e.g., ``This description is a hypothesis based on activation patterns, not a definitive explanation.''
\end{itemize}

\begin{figure}[ht]
    \centering
    \includegraphics[width=0.8\textwidth]{figures/intra_class_distance.png}
    \caption{Intra-class distance analysis showing how members of the same class may have different representations across layers, indicating potential concept fragmentation.}
    \label{fig:intra_class_distance}
\end{figure}

% Conclusion
\section{Conclusion}

Concept Trajectory Analysis (CTA) with LLM-powered interpretation represents a significant advancement in neural network interpretability, combining mathematical rigor with human-understandable explanations. Our work establishes a foundation for analyzing how concepts evolve and transform as they propagate through neural networks, providing insights into both the computational and semantic aspects of model behavior.

The integration of cross-layer metrics such as centroid similarity, membership overlap, and fragmentation scores provides a robust framework for quantifying concept evolution, while LLM-generated narratives translate these patterns into domain-meaningful explanations. Our experiments on traditional datasets (Titanic, Heart Disease) and comprehensive GPT-2 analysis demonstrate that this approach can identify nuanced decision processes, semantic organization patterns, and potential biases that might otherwise remain opaque.

The GPT-2 case studies reveal how transformer models systematically organize semantic knowledge, progressing from syntactic to semantic representations across layers. Our semantic pivot analysis shows how models handle contradictory information, while the semantic subtypes study demonstrates layer-specific clustering strategies for different conceptual categories. The per-layer ETS threshold optimization reveals that optimal clustering criteria vary systematically across network depth, providing new insights into transformer interpretability.

By formalizing the conditions under which activation-space clustering is valid and establishing methods to assess path stability, we have addressed critical gaps in the theoretical foundation of cluster-based interpretability. The incorporation of Explainable Threshold Similarity (ETS) further enhances transparency by providing verbalizably transparent membership conditions that can be directly communicated to domain experts.

This work represents a step toward bridging the divide between mathematical precision and human understanding in interpretability research, offering tools that can help researchers, developers, and end-users better understand the internal workings of neural networks. As models continue to grow in complexity and impact, approaches like CTA that combine quantitative analysis with qualitative explanation will become increasingly important for ensuring transparency, fairness, and trustworthiness in AI systems.

\begin{figure}[ht]
    \centering
    \includegraphics[width=0.8\textwidth]{figures/cluster_entropy.png}
    \caption{Cluster entropy analysis showing the distribution of information across the network's internal representation spaces.}
    \label{fig:cluster_entropy}
\end{figure}

% Future Directions
\section{Future Directions for Concept Trajectory Analysis}

Our discovery that GPT-2 organizes by grammatical function rather than semantic meaning opens revolutionary possibilities for interpretable AI. We outline key areas for advancing both the theoretical foundations and practical applications of CTA.

\subsection{Methodological Foundations}

\subsubsection{Advanced Metrics and Analysis}

\begin{itemize}
    \item \textbf{Inter-Cluster Path Density (ICPD)}: Develop metrics that analyze higher-order patterns in concept flow by examining multi-step transitions. ICPD could identify common patterns like return paths (where concepts temporarily diverge then reconverge) and similar-destination paths (reaching conceptually similar endpoints through different routes).
    
    \item \textbf{Path Interestingness Score}: Create composite metrics that combine transition rarity, similarity convergence, and coherence to automatically identify the most noteworthy paths for analysis. This would prioritize paths that reveal unexpected model behavior or critical decision points.
    
    \item \textbf{Feature Attribution for Transitions}: Integrate methods like Integrated Gradients or SHAP to understand which input features drive cluster transitions. For text, this could reveal which tokens cause semantic shifts; for medical data, which symptoms trigger risk reassessment.
\end{itemize}

\subsubsection{Enhanced Clustering Approaches}

\begin{itemize}
    \item \textbf{Explainable Threshold Similarity (ETS)}: Advance the implementation of ETS clustering \citep{kovalerchuk2024} to provide dimension-wise explanations for cluster membership. ETS declares activations similar if they differ by less than threshold $\tau_j$ in each dimension $j$, enabling transparent statements about cluster boundaries.
    
    \item \textbf{Hierarchical Clustering}: Develop multi-scale cluster structures where coarse clusters use loose thresholds and fine-grained subclusters use tighter bounds, enabling analysis at different levels of granularity.
    
    \item \textbf{Adaptive Thresholds}: Create methods to automatically determine optimal clustering thresholds per dimension based on activation distributions and downstream task requirements.
\end{itemize}

\subsubsection{Cluster Reproducibility and Validation}

\begin{itemize}
    \item \textbf{Cross-Architecture Stability}: Extend reproducibility analysis beyond training seeds to different model architectures, assessing whether discovered pathways represent fundamental computational patterns.
    
    \item \textbf{Statistical Significance Testing}: Develop rigorous statistical tests for pathway significance, distinguishing genuine organizational patterns from noise.
    
    \item \textbf{Causal Validation}: Use interventions and ablations to verify that discovered pathways causally influence model outputs rather than being mere correlations.
\end{itemize}

\subsubsection{Interactive Visualization Tools}

\begin{itemize}
    \item \textbf{Cluster Cards}: Develop interactive visualizations that summarize each cluster's properties, including representative examples, outliers, transition probabilities, and LLM-generated descriptions.
    
    \item \textbf{Real-Time Path Tracking}: Create lightweight tools for monitoring activation paths during inference, enabling debugging and analysis of specific model behaviors.
    
    \item \textbf{Comparative Visualization}: Build tools to compare pathways across different models, datasets, or time periods, revealing organizational differences and drift.
\end{itemize}

\subsection{Immediate Technical Improvements}

Building on the current implementation, several technical enhancements would strengthen CTA's rigor and applicability:

\subsubsection{Microcluster Lens Implementation}

\begin{itemize}
    \item \textbf{Hierarchical Sub-clustering}: Implement fine-grained analysis within highways to reveal semantic micro-organization. For instance, within the noun highway, identify sub-clusters for animate vs. inanimate entities, concrete vs. abstract concepts.
    
    \item \textbf{Adaptive Resolution}: Develop algorithms that automatically determine when to zoom into micro-clusters based on intra-cluster variance and task requirements.
    
    \item \textbf{Cross-Layer Micro-tracking}: Follow micro-cluster evolution to understand how fine-grained distinctions emerge, persist, or dissolve through network layers.
\end{itemize}

\subsubsection{Robustness and Validation}

\begin{itemize}
    \item \textbf{Cross-Seed Stability}: Run all experiments with multiple random seeds (N$\geq$5) to quantify variation in highway formation, cluster boundaries, and convergence rates. Report confidence intervals for all key metrics.
    
    \item \textbf{Clustering Quality Metrics}: Add silhouette scores, Davies-Bouldin indices, and Calinski-Harabasz scores to validate cluster coherence. Compare these across different k values to strengthen Gap statistic findings.
    
    \item \textbf{Inter-LLM Validation}: Use multiple LLMs (GPT-4, Claude, Gemini) for cluster interpretation and report agreement scores. Implement majority voting for final labels to reduce single-model bias.
    
    \item \textbf{Ablation Studies}: Systematically scramble POS tags, shuffle token positions, or randomize embeddings to verify that observed patterns disappear under null conditions, confirming they're not artifacts.
\end{itemize}

\subsubsection{Extended Analysis Capabilities}

\begin{itemize}
    \item \textbf{Multi-Token Context}: Extend beyond single-token analysis to study how context affects trajectories. Compare paths for "bank" in financial vs. river contexts, revealing context-dependent routing.
    
    \item \textbf{Training-Time Tracking}: Implement checkpointing to save activations at regular training intervals (every 1000 steps), enabling analysis of when grammatical organization emerges and how pathways form.
    
    \item \textbf{Quantitative Bias Metrics}: For medical AI, calculate demographic parity differences, equalized odds ratios, and disparate impact scores. Create pathway-based bias detection that identifies which neural routes exhibit unfair behavior.
\end{itemize}

\subsection{Advanced Applications for Language Models}

\subsubsection{Scaling to Complete Neural Cartography}

\begin{itemize}
    \item \textbf{Full Vocabulary Mapping}: Extend analysis from 1,228 words to entire vocabularies, revealing the complete ``highway system'' of neural language processing. We hypothesize discovering 50-100 major pathways handling different linguistic functions.
    
    \item \textbf{Compositional Analysis}: Study how models process bigrams, trigrams, and phrases to understand compositional meaning construction. Investigate whether multi-word expressions follow predictable combinations of single-word pathways.
    
    \item \textbf{Cross-Model Universal Patterns}: Map pathways across different model families (GPT, Claude, Gemini, LLaMA) to identify universal organizational principles versus architecture-specific patterns.
\end{itemize}

\subsubsection{Interpretable Pathways in Production}

\begin{itemize}
    \item \textbf{Real-Time Pathway Logging}: Implement efficient pathway tracking in production models with minimal computational overhead (<0.1\%), enabling models to access their own reasoning paths during generation.
    
    \item \textbf{Self-Debugging AI}: Enable models to detect and correct reasoning errors by examining pathway logs. For instance, if a financial term routes through an animal-related pathway, the model could recognize and correct the misrouting.
    
    \item \textbf{Pathway-Aware Generation}: Allow models to explicitly choose pathways based on task requirements—routing through logical reasoning pathways for mathematics or creative synthesis paths for storytelling.
\end{itemize}

\subsubsection{Meta-Analysis with Advanced Models}

\begin{itemize}
    \item \textbf{AI Understanding AI}: Use more powerful models to analyze millions of paths, stability metrics, and cluster patterns to discover organizational principles beyond human comprehension.
    
    \item \textbf{Automated Hypothesis Generation}: Employ LLMs to generate and test hypotheses about pathway formation, cluster evolution, and the emergence of grammatical organization.
    
    \item \textbf{Training Dynamics}: Study when and how grammatical organization emerges during training—does it appear suddenly at a phase transition or gradually evolve?
\end{itemize}

\subsection{Broader Impact and Applications}

\begin{itemize}
    \item \textbf{Interpretability-First Architecture}: Design new models with built-in pathway tracking and cluster organization, making interpretability a core feature rather than post-hoc analysis.
    
    \item \textbf{Beyond Language Models}: Extend CTA to vision transformers, multimodal models, and reinforcement learning agents to understand their organizational principles.
    
    \item \textbf{Real-Time Model Monitoring}: Deploy CTA in production to detect concept drift, identify emerging biases, and ensure models maintain expected organizational patterns.
    
    \item \textbf{Personalized Explanations}: Generate user-specific explanations by translating pathway information into conceptual frameworks appropriate for different expertise levels.
\end{itemize}

\subsection{Practical Use Cases}

\begin{itemize}
    \item \textbf{Prompt Strategy Evaluation}: Compare path density and fragmentation scores across prompt framings (e.g., Socratic vs. assertive) to reveal shifts in internal processing consistency.
    
    \item \textbf{Layerwise Ambiguity Detection}: Identify prompt-token pairs with divergent paths across layers, highlighting instability or multiple plausible interpretations.
    
    \item \textbf{Subgroup Drift Analysis}: Track membership overlap for datapoint groups (e.g., positive vs. negative sentiment) across layers to identify convergence patterns.
    
    \item \textbf{Behavioral Explanation}: Generate LLM-authored natural language summaries for archetypal paths, providing interpretable insights into model behavior.
    
    \item \textbf{Failure Mode Discovery}: Flag high-fragmentation paths as potential errors, misclassifications, or hallucinations.
    
    \item \textbf{Bias Detection}: Analyze paths for inputs with demographic markers to detect divergent behavior patterns that may indicate unfair treatment.
\end{itemize}

\subsection{Theoretical Advances}

\begin{itemize}
    \item \textbf{Mathematical Theory of Neural Organization}: Formalize why transformers converge to grammatical rather than semantic organization, potentially revealing fundamental principles of efficient information processing.
    
    \item \textbf{Optimal Pathway Design}: Develop theory for designing optimal pathway structures for specific tasks, moving from emergent to engineered organization.
    
    \item \textbf{Cross-Domain Transfer}: Understand how pathway structures enable or inhibit transfer learning, using CTA to optimize model adaptation.
\end{itemize}

As we advance these techniques, we envision a future where neural network interpretability becomes as routine and insightful as medical imaging. The discovery that language models organize grammatically rather than semantically is just the beginning—the full cartography of neural organization awaits exploration.

% Acknowledgments
\section*{Acknowledgments}
This work was created as part of an exploration of interpretability methods for neural networks. We appreciate the open-source community for developing the libraries and tools used in this research.

% References
\bibliographystyle{plainnat}
\bibliography{references}

\end{document}